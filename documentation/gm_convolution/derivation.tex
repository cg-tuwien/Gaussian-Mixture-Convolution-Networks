% ---------------------------------------------------------------------------
\documentclass{paper}

\usepackage[T1]{fontenc}

%\usepackage{cite}  % comment out for biblatex with backend=biber 
% ---------------------------



\usepackage[utf8]{inputenc}
\usepackage{amsmath}
\usepackage{amssymb}
\usepackage{amsthm}
\usepackage{supertabular}
%\usepackage{cite}
%\usepackage[utf8]{inputenc}
\usepackage[font={it},labelfont={bf,up}]{caption}
\usepackage[font={small,it}]{subcaption}
\usepackage{hyperref}
\hypersetup{
	colorlinks=true,
	linkcolor=blue,
	filecolor=magenta,      
	urlcolor=blue,
	citecolor=magenta,
}

\newtheorem{theorem}{Theorem}
% end of prologue

\newcommand{\todo}[1]{{\color{red}\textbf{(TODO: {#1})}}} % Comment for the final version, to raise errors.
\newcommand{\MW}[1]{{\color[rgb]{0,0.7,0}\textbf{(MW: {#1})}}} % Comment for the final version, to raise errors.
\newcommand{\AC}[1]{{\color{magenta}{#1}}} % Comment for the final version, to raise errors.
\newcommand{\abs}[1]{\left| #1 \right|}
\newcommand{\tAbs}[1]{| #1 |}
\newcommand{\F}{\ensuremath{\mathcal{F}}}
\newcommand{\vr}[1]{\ensuremath{\boldsymbol{#1}}}
\newcommand{\tr}[1]{\ensuremath{\boldsymbol{#1}}}
\newcommand{\T}[1]{\ensuremath{{#1}^T}}
\newcommand{\f}[1]{\operatorname{#1}}

\newcommand{\alphavec}[0]{\ensuremath{\vr{\alpha{}}}}
\newcommand{\betavec}[0]{\ensuremath{\vr{\beta{}}}}
\newcommand{\omegavec}[0]{\ensuremath{\vr{\omega{}}}}
\newcommand{\xivec}[0]{\ensuremath{\vr{\xi{}}}}
\newcommand{\avec}[0]{\ensuremath{\vr{a}}}
\newcommand{\bvec}[0]{\ensuremath{\vr{b}}}
\newcommand{\xvec}[0]{\ensuremath{\vr{x}}}
\newcommand{\yvec}[0]{\ensuremath{\vr{y}}}
\newcommand{\zvec}[0]{\ensuremath{\vr{z}}}

\newcommand{\Ctns}[0]{\ensuremath{\tr{C}}}
\newcommand{\Gammatns}[0]{\ensuremath{\tr{\Gamma}}}

\newcount\colveccount
\newcommand*\colvec[1]{
	\global\colveccount#1
	\begin{pmatrix}
		\colvecnext
	}
	\def\colvecnext#1{
		#1
		\global\advance\colveccount-1
		\ifnum\colveccount>0
		\\
		\expandafter\colvecnext
		\else
	\end{pmatrix}
	\fi
}

% ---------------------------------------------------------------------

\title{Fourier Transform and Convolution of Gaussian Mixtures}
\author{Adam Celarek\\Research Unit of Computer Graphics, TU Wien}

\begin{document}
\maketitle

%-------------------------------------------------------------------------
\section{Definitions}
\subsection*{Notation}
\begin{center}
	\begin{supertabular}{rp{8cm}}
		$a$	& Scalar \\
		$\vr{a}$	& Column vector \\
		$A$			& Matrix \\ 
		$\tr{A}$	& Tensor/Array with more than 2 dimensions, e.g. $\tr{A} \in \mathbb{R}^{L \times M \times N}$ \\ 
		$\tr{A}_{\star, 3, 2}$
					& Element of an array.
					The first index is the row, the second a column and the third a slice.
					"$\star$" selects all elements in that dimension.
					So in this case the full 3rd column of the second slice. \\ 
		$A^T, \vr{a}^T$
					& Transpose of $A$ and $\vec{a}$ (row vector) \\
		$i$			& Imaginary unit \\
		$\vr{x}, \vr{\omega}$
					& Spatial (time) domain and frequency domain coordinates \\
		$\f{f}(\vr{x})$
					& Function in spatial space \\
		$\f{F}(\vr{\omega})$
					& Function in Fourier space \\
		$\f{F} = \F \f{f}$
					& Fourier transform \\
		$\f{f} = \F^{-1} \f{F}$
					& Inverse Fourier transform \\
	\end{supertabular}
\end{center}
Zero based indices are used and summation ends at $N-1$.

\subsection*{Gaussian function and mixture model}
We define the multi dimensional Gaussian function as
\begin{align}
\label{eq:gaussian_definition}
\f{g}(\vr{x}, a, \vr{b}, C) = a e^{-\frac{1}{2}(\vr{x}-\vr{b})^TC^{-1}(\vr{x}-\vr{b})},
\end{align}
where $a$ is the height, $\vr{b}$ the shift, $C$ the shape (covariance) of the Gaussian, and all parameters are real.
The factor $\frac{1}{2}$ and the inversion of $C$ are there to make the derivation and subsequent formulae simpler.
This Gaussian would turn into a normal distribution if $a$ were replaced with the inverse of the integral of $e^{-\frac{1}{2}...}$.

A Gaussian mixture is defined as
\begin{align}
	\f{gm}(\vr{x}, \vr{a}, B, \tr{C}) = \sum_{n=0}^{N} \f{g}(\vr{x}, \vr{a}_n, B_{\star, n}, C_{\star, \star, n}),
\end{align}
where $N$ is the number of Gaussians.
Our definition does not include any normalisation.
If desired, this could be included in $\vr{a}$.

We also know the integral of a single dimensional Gaussian \cite{wiki_gaussian_int}:
\begin{align}
	\label{eq:gaussian_integral}
	\int_{\mathbb{R}} a e^{-(x+b)^2 c} dx = a\sqrt{\frac{\pi}{c}}
\end{align}

\subsection*{Fourier transformation and its inverse}
The definition of the Fourier transform is taken from Osgood's lecture notes (be careful, there is the whole book \cite{osgood_book} and separately the corrected version of chapter 8 \cite{osgood_chapter8}, which is the important one for us ):

\begin{align}
	\f{F} (\vr{\omega}) = \F \f{f} (\vr{\omega}) &= \int_{\mathbb{R}^n} \f{f}(\vr{x}) e^{-2 \pi i \vr{\omega}^T \vr{x}} d\vr{x} \\
	\f{f} (\vr{x}) = (\F^{-1} \f{F}) (\vr{x}) &= \int_{\mathbb{R}^n} \f{F}(\vr{\omega}) e^{2 \pi i \vr{\omega}^T \vr{x}} d\vr{x}
\end{align}

Properties are:
\begin{itemize}
	\item Scalar linearity: \\
	      $\F(\alpha \f{f} + \beta \f{g})(\omegavec) = \alpha \F \f{f}(\omega) + \beta \F \f{g}(\omegavec)$
	\item Shift / Translation in spatial domain: \\
	      $\F\f{f}(\xvec + \avec)(\omegavec) = e^{2 \pi i \avec^T \omegavec} (\F \f{f})(\omegavec)$
	\item General stretch / Transformation in spatial domain:\\
		  $\F\f{f}(A\vr{x})(\omegavec) = \frac{1}{\abs{\det(A)}} \F \f{f}(\xvec)((A^{-1})^T\omegavec)$
	\item Convolution:\\
		  $\F(\f{f}\ast\f{g})(\omegavec) = \F\f{f}(\omegavec) \F\f{g}(\omegavec)$ and $\F(\f{f}\f{g})(\omegavec) = (\F\f{f} \ast \F\f{g})(\omegavec)$
\end{itemize}

\clearpage
\section{Fourier transform of a Gaussian}
\begin{theorem}
	The Fourier transform of a multi-dimensional Gaussian is
	\begin{align}
	\label{eq:gaussian_fourier_transform}
	(\F\f{g}(\vr{x}, a, \bvec, C))(\omegavec) = a \sqrt{(2 \pi)^\eta \f{det}(C)} e^{-2 \pi^2 \omegavec^T C \omegavec - 2\pi i \bvec^T \omegavec},
	\end{align}
	where $\eta$ is the number of dimensions.
\end{theorem}
Because the result is in the complex domain it is not generally a Gaussian, at least not by our definition in Equation \ref{eq:gaussian_definition}.
It becomes Gaussian when the position in spatial domain (shift $\bvec$) is zero.
Moreover, its absolute value is Gaussian (without proof), but that is not interesting in this context.

\begin{proof} (lengthy, until the end of the section)
\subsection*{Simplification}
We defined a Gaussian as 
\begin{align*}
	\f{g}(\vr{x}, a, \vr{b}, C) = a e^{-\frac{1}{2}(\vr{x}-\vr{b})^T C^{-1}(\vr{x}-\vr{b})},
\end{align*}
but we will ignore the factor $a$ and the shift $\vr{b}$ for now.
These can be reintroduced later via the scalar linearity and shift properties of the Fourier transform.
For a simpler notation we substitute
\begin{align*}
	\xi = -2 \pi i \omegavec.
\end{align*}
Therefore we start with
\begin{align*}
	(\F\f{g}(\vr{x}, C))(\frac{\xi}{-2 \pi i}) &= \int_{\mathbb{R}^n} e^{-\frac{1}{2} \vr{x}^T C^{-1} \vr{x}} e^{\vr{\xi}^T \vr{x}} d\vr{x} \\
	&= \int_{\mathbb{R}^n} e^{-\frac{1}{2} \vr{x}^T C^{-1} \vr{x} + \vr{\xi}^T \vr{x}} d\vr{x}.
\end{align*}

\subsection*{Moving the linear term out of the integral}
We substitute
\begin{align*}
	\vr{x} = \varphi(\vr{y}) &= \vr{y} + C\vr{\xi} \\
	\Bigg( \f{D}\varphi(\yvec) &= \frac{\partial \vr{y} + C\vr{\xi}}{\partial \vr{y}} = I \Bigg)
\end{align*}
Resulting in
\begin{align*}
	(\F\f{g}(\vr{x}, C))(\frac{\xi}{-2 \pi i})
	&= \int_{\mathbb{R}^n} e^{-\frac{1}{2} (\yvec + C\xivec)^T C^{-1} (\yvec + C\xivec) + \xivec^T (\yvec + C\xivec)} \abs{\f{det}(\f{D}\varphi)} d\yvec.
\end{align*}
$\abs{\f{det}(\f{D}\varphi)}$ is 1, and we can reduce the exponent:
\begin{align*}
	&\quad -\frac{1}{2} (\yvec + C\xivec)^T C^{-1} (\yvec + C\xivec) + \xivec^T (\yvec + C\xivec) = \\
	&= -\frac{1}{2} \left( (\yvec^T C^{-1} + \xivec^T)(\yvec + C \xivec) \right) + \xivec^T \yvec + \xivec^T C \xivec = \\
	&= -\frac{1}{2} \yvec^T C^{-1} \yvec - \frac{1}{2} \yvec^T \xivec -\frac{1}{2} \xivec^T \yvec -\frac{1}{2} \xivec^T C \xivec + \xivec^T \yvec + \xivec^T C \xivec = \\
	&= -\frac{1}{2} \yvec^T C^{-1} \yvec + \frac{1}{2} \xivec^T C \xivec.
\end{align*}
using the fact that C is symmetric.
Putting it back results in
\begin{align}
	\label{eq:fourier_only_gauss_int_left}
	(\F\f{g}(\vr{x}, C))(\frac{\xi}{-2 \pi i})
	&= e^{\frac{1}{2} \xivec^T C \xivec} \int_{\mathbb{R}^n} e^{-\frac{1}{2} \yvec^T C^{-1} \yvec}  d\yvec.
\end{align}
Nice, so we were able to move that $\xivec$ term out of the integral.
Now we have to evaluate the latter.
\subsection*{Computing the integral of the quadratic term}
For that we use the eigendecomposition of $C$ with orthonormal matrices $Q$ and diagonal eigenvalue matrix $\Lambda$ \cite{wiki_eigendecomp}:
\begin{align*}
	C^{-1} &= (Q \Lambda Q^T)^{-1} = Q \Lambda^{-1} Q^T, \nonumber \\
	\int_{\mathbb{R}^n} e^{-\frac{1}{2} \yvec^T C^{-1} \yvec}  d\yvec
	&= \int_{\mathbb{R}^n} e^{-\frac{1}{2} \yvec^T Q \Lambda^{-1} Q^T \yvec}  d\yvec =  \nonumber \\
	&= \int_{\mathbb{R}^n} e^{-\frac{1}{2} (Q^T \yvec)^T \Lambda^{-1} (Q^T \yvec)}  d\yvec
\end{align*}
and substituting
\begin{align*}
\zvec = \psi(\yvec) &= Q^T \yvec \\
\Bigg( \f{D}\psi(\yvec) &= \frac{\partial Q^T \vr{x}}{\partial \vr{x}} = Q^T \Bigg)
\end{align*}
gives (the determinant is 1 again due to orthonormality of Q)
\begin{align*}
	&\quad \int_{\mathbb{R}^n} e^{-\frac{1}{2} \zvec^T \Lambda^{-1} \zvec} \abs{\f{det}(\f{D}\psi)} d\zvec =\\
	&= \int_{\mathbb{R}^n} e^{-\frac{1}{2} \zvec^T \Lambda^{-1} \zvec} d\zvec =\\
	&= \int_{\mathbb{R}^n} e^{-\frac{1}{2} \sum_{k=0}^\eta \zvec_k^2 \Lambda_{kk}^{-1} } d\zvec =\\
	&= \int_{\mathbb{R}} e^{-\frac{1}{2} \zvec_1^2 \Lambda_{11}^{-1} } \int_{\mathbb{R}} e^{-\frac{1}{2} \zvec_2^2 \Lambda_{22}^{-1} } \dots \int_{\mathbb{R}} e^{-\frac{1}{2} \zvec_\eta^2 \Lambda_{\eta\eta}^{-1} } d\zvec_\eta d\zvec_{\eta-1} \dots d\zvec_1 =\\
	&= \prod_{k=0}^\eta \sqrt{2 \pi \Lambda_{kk}} = \sqrt{(2 \pi)^\eta \f{det}(C)},
\end{align*}
where $\eta$ is the number of dimensions, and we use Equation \ref{eq:gaussian_integral} and the fact that the determinant is equal to the product of the eigenvalues. To summarise
\begin{align}
	\int_{\mathbb{R}^n} e^{-\frac{1}{2} \yvec^T C^{-1} \yvec}  d\yvec = \sqrt{(2 \pi)^\eta \f{det}(C)}.
\end{align}

\subsection*{Putting it back together}
Plugging that back into Equation \ref{eq:fourier_only_gauss_int_left} gives us
\begin{align*}
	(\F\f{g}(\vr{x}, C))(\frac{\xi}{-2 \pi i})
	&= e^{\frac{1}{2} \xivec^T C \xivec} \int_{\mathbb{R}^n} e^{-\frac{1}{2} \yvec^T C^{-1} \yvec}  d\yvec = \\
	&= e^{\frac{1}{2} \xivec^T C \xivec} \sqrt{(2 \pi)^\eta \f{det}(C)}
\end{align*}
and, after recovering $\omega$ and friends
\begin{align*}
(\F\f{g}(\vr{x}, C))(\omegavec)
&= e^{\frac{1}{2} (-2 \pi i \omegavec)^T C (-2 \pi i \omegavec)} \sqrt{(2 \pi)^\eta \f{det}(C)} = \\
&= \sqrt{(2 \pi)^\eta \f{det}(C)} e^{-2 \pi^2 \omegavec^T C \omegavec}.
\end{align*}
Finally, we use the shift and scalar linearity properties of the Fourier transform:
\begin{align}
(\F\f{g}(\vr{x}, a, \bvec, C))(\omegavec) &= a (\F\f{g}(\vr{x} - \bvec, C))(\omegavec) = \nonumber \\
&= a \sqrt{(2 \pi)^\eta \f{det}(C)} e^{-2\pi i \bvec^T \omegavec} e^{-2 \pi^2 \omegavec^T C \omegavec} =\nonumber \\
&= a \sqrt{(2 \pi)^\eta \f{det}(C)} e^{-2 \pi^2 \omegavec^T C \omegavec - 2\pi i \bvec^T \omegavec}.
\end{align}

\end{proof}

\clearpage
\section{Convolution}
\label{sec:convolution}
\begin{theorem}
	\begin{align}
		\f{g}(\xvec, a, \bvec, C) \ast \f{g}(\xvec, \alpha, \vr{\beta}, \Gamma) = 
		\f{g}\left(\xvec, \frac{a \alpha \sqrt{(2 \pi)^\eta} \sqrt{\f{det}(C\Gamma)}}{\sqrt{\f{det}(C+\Gamma)}}, \bvec + \betavec, C + \Gamma\right),
	\end{align}
	where $\eta$ is the number of dimensions.
\end{theorem}
\begin{proof}
Convolution in spatial domain is multiplication in Fourier domain.
We are interested in
\begin{align*}
	\f{g}(\xvec, a, \bvec, C) \ast \f{g}(\xvec, \alpha, \vr{\beta}, \Gamma),
\end{align*}
and therefore in
\begin{align*}
	&\ \quad \F^{-1}(\F\f{g}(\xvec, a, \bvec, C) \F\f{g}(\xvec, \alpha, \vr{\beta}, \Gamma))(\xvec) = \nonumber \\
	&= \F^{-1}\left(a \sqrt{(2 \pi)^\eta \f{det}(C)}      e^{-2 \pi^2 \omegavec^T C \omegavec      - 2\pi i \bvec^T \omegavec}
	         \alpha \sqrt{(2 \pi)^\eta \f{det}(\Gamma)} e^{-2 \pi^2 \omegavec^T \Gamma \omegavec - 2\pi i \betavec^T \omegavec}\right) = \nonumber \\
	&= \F^{-1}\left(a \alpha (2 \pi)^\eta \sqrt{\f{det}(C) \f{det}(\Gamma)}
			        e^{-2 \pi^2 \omegavec^T C \omegavec - 2\pi i \bvec^T \omegavec - 2 \pi^2 \omegavec^T \Gamma \omegavec - 2\pi i \betavec^T \omegavec}\right) = \nonumber \\
	\label{eq:gaussian_convolution_before_inverse_transform}
	&= \F^{-1}\left(a \alpha (2 \pi)^\eta \sqrt{\f{det}(C\Gamma)}
                    e^{-2 \pi^2 \omegavec^T (C + \Gamma) \omegavec - 2\pi i (\bvec + \betavec)^T \omegavec}\right).
\end{align*}
This looks similar to Equation \ref{eq:gaussian_fourier_transform} and we can guess the inverse transform:
\begin{align*}
	&\ \quad \frac{(2 \pi)^\eta \sqrt{\f{det}(C\Gamma)}}{\sqrt{(2\pi)^\eta \f{det}(C+\Gamma)}}
	        \F^{-1}\left(
	                     a \alpha \sqrt{(2\pi)^\eta \f{det}(C+\Gamma)}
	                     e^{-2 \pi^2 \omegavec^T (C + \Gamma) \omegavec - 2\pi i (\bvec + \betavec)^T \omegavec}\right) = \\
	&= \frac{\sqrt{(2 \pi)^\eta} \sqrt{\f{det}(C\Gamma)}}{\sqrt{\f{det}(C+\Gamma)}}
	   \f{g}(\xvec, a \alpha, \bvec + \betavec, C + \Gamma)
\end{align*}
\end{proof}

\subsection{Numerical behaviour}
There are no surprises with the Gaussian itself, but $\frac{\sqrt{\f{det}(C\Gamma)}}{\sqrt{\f{det}(C+\Gamma)}}$ looks a bit scary.
In particular, it is undefined iff both covariances have a determinant of 0.
This is equivalent with the statement that they are not full rank, or that each of them has at least one eigenvalue equal to 0.
Graphically this would mean, that the Gaussians have zero variance in at least one direction.

\begin{theorem}
	In case that at least one covariance has a determinant greater 0, then the factor is bounded by $\sqrt{\f{min}(\f{det}(C), \f{det}(\Gamma))}$.
\end{theorem}
\begin{proof}
Since $\f{det}(C+\Gamma) \ge \f{det}(C) + \f{det}(\Gamma)$ \cite{wiki_det_inequalities}, and assume wlog $\f{det}(C) = \f{min}(\f{det}(C), \f{det}(\Gamma))$:
\begin{align*}
	\frac{\sqrt{\f{det}(C\Gamma)}}{\sqrt{\f{det}(C+\Gamma)}} \le \frac{\sqrt{\f{det}(C\Gamma)}}{\sqrt{\f{det}(C) + \f{det}(\Gamma)}} &\le \sqrt{\f{min}(\f{det}(C), \f{det}(\Gamma))} \\
	\frac{\f{det}(C\Gamma)}{\f{det}(C) + \f{det}(\Gamma)} &\le \f{det}(C) \\
	\f{det}(C)\f{det}(\Gamma) &\le \f{det}(C)\f{det}(C) + \f{det}(C)\f{det}(\Gamma) \\
	0 &\le \f{det}(C)^2
\end{align*}
The last statement is true because both $C$ and $\Gamma$ are positive definite.
\end{proof}

\section{Convolution of Gaussian mixtures}
Convolution of 2 Gaussian mixtures is relatively straight forward as both, the Fourier transform and its inverse, are linear, and therefore it is possible to apply convolution on every Gaussian pair separately.
Convolving a Gaussian mixture with $n$ terms with another Gaussian mixture with $m$ terms results in a mixture with $nm$ terms.

\begin{align*}
	&\ \quad \f{gm}(\xvec, \avec, B, \Ctns) \ast \f{gm}(\xvec, \alphavec, \Pi, \Gammatns) = \\
	&= \F^{-1}\left(\F\f{gm}(\xvec, \avec, B, \Ctns) \F\f{gm}(\xvec, \alphavec, \Pi, \Gammatns)\right) = \\
	&= \F^{-1}\left(\sum_{m}\F\f{g}(\xvec, \avec_m, B_{\star, m}, \Ctns_{\star, \star, m}) \sum_n \F\f{g}(\xvec, \alphavec_n, \Pi_{\star, n}, \Gammatns_{\star, \star, n})\right) = \\
	&= \F^{-1}\left(\sum_{m}\sum_n  \F\f{g}(\xvec, \avec_m, B_{\star, m}, \Ctns_{\star, \star, m}) \F\f{g}(\xvec, \alphavec_n, \Pi_{\star, n}, \Gammatns_{\star, \star, n})\right) = \\
	&= \sum_{m}\sum_n \f{g}(\xvec, \avec_m, B_{\star, m}, \Ctns_{\star, \star, m}) \ast \f{g}(\xvec, \alphavec_n, \Pi_{\star, n}, \Gammatns_{\star, \star, n}) = \\
	&= \sum_{m}\sum_n \f{g}\left(\xvec, \frac{\avec_m \alphavec_n \sqrt{(2 \pi)^\eta} \sqrt{\f{det}(\Ctns_{\star, \star, m}\Gammatns_{\star, \star, n})}}{\sqrt{\f{det}(\Ctns_{\star, \star, m}+\Gammatns_{\star, \star, n})}}, B_{\star, m} + \Pi_{\star, n}, \Ctns_{\star, \star, m} + \Gammatns_{\star, \star, n}\right)
\end{align*}

\begin{thebibliography}{9}% 2nd arg is the width of the widest label.
	\bibitem{wiki_gaussian_int} \url{https://en.wikipedia.org/w/index.php?title=Gaussian_integral&oldid=920544461}
	\bibitem{wiki_eigendecomp} \url{https://en.wikipedia.org/wiki/Eigendecomposition_of_a_matrix#Eigendecomposition_of_a_matrix}
	\bibitem{wiki_det_inequalities} \url{https://en.wikipedia.org/w/index.php?title=Determinant&oldid=922725964#Properties_of_the_determinant}
	\bibitem{osgood_chapter8} \url{https://see.stanford.edu/materials/lsoftaee261/chap8.pdf}
	\bibitem{osgood_book} \url{https://see.stanford.edu/materials/lsoftaee261/book-fall-07.pdf}
\end{thebibliography}

\end{document}
