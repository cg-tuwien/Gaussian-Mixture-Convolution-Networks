% ---------------------------------------------------------------------------
\documentclass{paper}

\usepackage[T1]{fontenc}

%\usepackage{cite}  % comment out for biblatex with backend=biber 
% ---------------------------



\usepackage[utf8]{inputenc}
\usepackage{amsmath}
\usepackage{amssymb}
\usepackage{amsthm}
\usepackage{supertabular}
%\usepackage{cite}
%\usepackage[utf8]{inputenc}
\usepackage[font={it},labelfont={bf,up}]{caption}
\usepackage[font={small,it}]{subcaption}
\usepackage{hyperref}
\hypersetup{
	colorlinks=true,
	linkcolor=blue,
	filecolor=magenta,      
	urlcolor=blue,
	citecolor=magenta,
}

\newtheorem{theorem}{Theorem}
% end of prologue

\newcommand{\todo}[1]{{\color{red}\textbf{(TODO: {#1})}}} % Comment for the final version, to raise errors.
\newcommand{\MW}[1]{{\color[rgb]{0,0.7,0}\textbf{(MW: {#1})}}} % Comment for the final version, to raise errors.
\newcommand{\AC}[1]{{\color{magenta}{#1}}} % Comment for the final version, to raise errors.
\newcommand{\abs}[1]{\left| #1 \right|}
\newcommand{\tAbs}[1]{| #1 |}
\newcommand{\F}{\ensuremath{\mathcal{F}}}
\newcommand{\vr}[1]{\ensuremath{\boldsymbol{#1}}}
\newcommand{\tr}[1]{\ensuremath{\boldsymbol{#1}}}
\newcommand{\T}[1]{\ensuremath{{#1}^T}}
\newcommand{\f}[1]{\operatorname{#1}}
\newcommand*\diff{\mathop{}\!\mathrm{d}}

\newcommand{\alphavec}[0]{\ensuremath{\vr{\alpha{}}}}
\newcommand{\betavec}[0]{\ensuremath{\vr{\beta{}}}}
\newcommand{\omegavec}[0]{\ensuremath{\vr{\omega{}}}}
\newcommand{\xivec}[0]{\ensuremath{\vr{\xi{}}}}
\newcommand{\avec}[0]{\ensuremath{\vr{a}}}
\newcommand{\bvec}[0]{\ensuremath{\vr{b}}}
\newcommand{\xvec}[0]{\ensuremath{\vr{x}}}
\newcommand{\yvec}[0]{\ensuremath{\vr{y}}}
\newcommand{\zvec}[0]{\ensuremath{\vr{z}}}

\newcommand{\Ctns}[0]{\ensuremath{\tr{C}}}
\newcommand{\Gammatns}[0]{\ensuremath{\tr{\Gamma}}}

\newcount\colveccount
\newcommand*\colvec[1]{
	\global\colveccount#1
	\begin{pmatrix}
		\colvecnext
	}
	\def\colvecnext#1{
		#1
		\global\advance\colveccount-1
		\ifnum\colveccount>0
		\\
		\expandafter\colvecnext
		\else
	\end{pmatrix}
	\fi
}

% ---------------------------------------------------------------------

\title{Integrals of Gaussian Mixtures}
%\author{Adam Celarek\\Research Unit of Computer Graphics, TU Wien}
\author{Anonymous}

\begin{document}
\maketitle

%-------------------------------------------------------------------------
\section{Definitions}
\subsection*{Notation}
\begin{center}
	\begin{supertabular}{rp{8cm}}
		$a$	& Scalar \\
		$\vr{a}$	& Column vector \\
		$A$			& Matrix \\ 
		$\tr{A}$	& Tensor/Array with more than 2 dimensions, e.g. $\tr{A} \in \mathbb{R}^{L \times M \times N}$ \\ 
		$\tr{A}_{\star, 3, 2}$
					& Element of an array.
					The first index is the row, the second a column and the third a slice.
					"$\star$" selects all elements in that dimension.
					So in this case the full 3rd column of the second slice. \\ 
		$A^T, \vr{a}^T$
					& Transpose of $A$ and $\vec{a}$ (row vector) \\
		$i$			& Imaginary unit \\
		$\vr{x}, \vr{\omega}$
					& Spatial (time) domain and frequency domain coordinates \\
		$\f{f}(\vr{x})$
					& Function in spatial space \\
		$\f{F}(\vr{\omega})$
					& Function in Fourier space \\
		$\f{F} = \F \f{f}$
					& Fourier transform \\
		$\f{f} = \F^{-1} \f{F}$
					& Inverse Fourier transform \\
	\end{supertabular}
\end{center}
Zero based indices are used and summation ends at $N-1$.

\subsection*{Gaussian function and mixture model}
We define the multi dimensional Gaussian function as
\begin{align}
\label{eq:gaussian_definition}
\f{g}(\vr{x}, a, \vr{b}, C) = a e^{-\frac{1}{2}(\vr{x}-\vr{b})^TC^{-1}(\vr{x}-\vr{b})},
\end{align}
where $a$ is the height, $\vr{b}$ the shift, $C$ the shape (covariance) of the Gaussian, and all parameters are real.
The factor $\frac{1}{2}$ and the inversion of $C$ are there to make convolution simpler.
This Gaussian would turn into a normal distribution if $a$ were replaced with the inverse of the integral of $e^{-\frac{1}{2}...}$.

A Gaussian mixture is defined as
\begin{align}
	\f{gm}(\vr{x}, \vr{a}, B, \tr{C}) = \sum_{n=0}^{N} \f{g}(\vr{x}, \vr{a}_n, B_{\star, n}, C_{\star, \star, n}),
\end{align}
where $N$ is the number of Gaussians.
Our definition does not include any normalisation.
If desired, this could be included in $\vr{a}$.


\section{Integral}
According to wikipedia\cite{wiki_simple_gaussian_int}
\begin{align}
	\label{eq: simple_guassian_int}
	\int_{\mathbb{R}^n} e^{-\vr{x}^TA\vr{x}} \diff \vr{x} = \sqrt{\frac{\pi^n}{\det{A}}},
\end{align}
and\cite{wiki_det_inverse}
\begin{align}
\det{A^{-1}} &= \frac{1}{\det{A}}, \text{ and}\\
\det{cA} &= c^n \det{A}
\end{align}
for an $n \times n$ matrix.

In our Gaussian definition\ref{eq:gaussian_definition} we use an inverted covariance matrix, so Equation \ref{eq: simple_guassian_int} becomes
\begin{align}
	\int_{\mathbb{R}^n} e^{-\vr{x}^TA^{-1}\vr{x}} \diff \vr{x} = \sqrt{\pi^n \det{A}}.
\end{align}

The position shift $\vr{b}$ has no effect on the integral, therefore our result for a single Gaussian is

\begin{align}
	\int_{\mathbb{R}^n} a e^{-\frac{1}{2}(\vr{x}-\vr{b})^TC^{-1}(\vr{x}-\vr{b})} \diff \vr{x} &= a \int_{\mathbb{R}^n} e^{-\frac{1}{2}\vr{x}^TC^{-1}\vr{x}} \diff \vr{x} = a \sqrt{\pi^n \det{2 C}} \\
	&= a \sqrt{(2 \pi)^n \det{C}}.
\end{align}

For a Gaussian mixture it is
\begin{align}
	\int_{\mathbb{R}^n} \f{gm}(\vr{x}, \vr{a}, B, \tr{C}) \diff \vr{x} &= \int_{\mathbb{R}^n} \sum_{m=0}^{M} \f{g}(\vr{x}, \vr{a}_m, B_{\star, m}, C_{\star, \star, m}) \diff \vr{x} \\
	&= \sum_{m=0}^{M} a_m \sqrt{(2 \pi)^n \det{C_{\star, \star, m}}}.
\end{align}


\begin{thebibliography}{9}% 2nd arg is the width of the widest label.
	\bibitem{wiki_simple_gaussian_int} \url{https://en.wikipedia.org/w/index.php?title=Gaussian_function&oldid=931706640#Multi-dimensional_Gaussian_function}
	\bibitem{wiki_det_inverse} \url{https://en.wikipedia.org/w/index.php?title=Determinant&oldid=930476527#Properties_of_the_determinant}
\end{thebibliography}

\end{document}
