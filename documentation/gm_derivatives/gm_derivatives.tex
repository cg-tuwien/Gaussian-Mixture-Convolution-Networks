% ---------------------------------------------------------------------------
\documentclass{paper}

\usepackage[T1]{fontenc}

%\usepackage{cite}  % comment out for biblatex with backend=biber 
% ---------------------------



\usepackage[utf8]{inputenc}
\usepackage{amsmath}
\usepackage{amssymb}
\usepackage{amsthm}
\usepackage{supertabular}
%\usepackage{cite}
%\usepackage[utf8]{inputenc}
\usepackage[font={it},labelfont={bf,up}]{caption}
\usepackage[font={small,it}]{subcaption}
\usepackage{hyperref}
\hypersetup{
	colorlinks=true,
	linkcolor=blue,
	filecolor=magenta,      
	urlcolor=blue,
	citecolor=magenta,
}

\newtheorem{theorem}{Theorem}
% end of prologue

\newcommand{\todo}[1]{{\color{red}\textbf{(TODO: {#1})}}} % Comment for the final version, to raise errors.
\newcommand{\MW}[1]{{\color[rgb]{0,0.7,0}\textbf{(MW: {#1})}}} % Comment for the final version, to raise errors.
\newcommand{\AC}[1]{{\color{magenta}{#1}}} % Comment for the final version, to raise errors.
\newcommand{\abs}[1]{\left| #1 \right|}
\newcommand{\tAbs}[1]{| #1 |}
\newcommand{\F}{\ensuremath{\mathcal{F}}}
\newcommand{\vr}[1]{\ensuremath{\boldsymbol{#1}}}
\newcommand{\mx}[1]{\ensuremath{#1}}
\newcommand{\tr}[1]{\ensuremath{\boldsymbol{#1}}}
\newcommand{\T}[1]{\ensuremath{{#1}^T}}
\newcommand{\f}[1]{\operatorname{#1}}
\newcommand*\diff{\mathop{}\!\mathrm{d}}

\newcommand{\alphavec}[0]{\ensuremath{\vr{\alpha{}}}}
\newcommand{\betavec}[0]{\ensuremath{\vr{\beta{}}}}
\newcommand{\omegavec}[0]{\ensuremath{\vr{\omega{}}}}
\newcommand{\xivec}[0]{\ensuremath{\vr{\xi{}}}}
\newcommand{\avec}[0]{\ensuremath{\vr{a}}}
\newcommand{\bvec}[0]{\ensuremath{\vr{b}}}
\newcommand{\xvec}[0]{\ensuremath{\vr{x}}}
\newcommand{\yvec}[0]{\ensuremath{\vr{y}}}
\newcommand{\zvec}[0]{\ensuremath{\vr{z}}}

\newcommand{\Ctns}[0]{\ensuremath{\tr{C}}}
\newcommand{\Gammatns}[0]{\ensuremath{\tr{\Gamma}}}

\newcount\colveccount
\newcommand*\colvec[1]{
	\global\colveccount#1
	\begin{pmatrix}
		\colvecnext
	}
	\def\colvecnext#1{
		#1
		\global\advance\colveccount-1
		\ifnum\colveccount>0
		\\
		\expandafter\colvecnext
		\else
	\end{pmatrix}
	\fi
}

% ---------------------------------------------------------------------

\title{Derivatives of Gaussian Mixtures with respect to \vr{x} and the parameters}
\author{Adam Celarek\\Research Unit of Computer Graphics, TU Wien}

\begin{document}
\maketitle

%-------------------------------------------------------------------------
\section{Definitions}
\subsection*{Notation}
\begin{center}
	\begin{supertabular}{rp{8cm}}
		$a$	& Scalar \\
		$\vr{a}$	& Column vector \\
		$\mx{A}$			& Matrix \\ 
		$\tr{A}$	& Tensor/Array with more than 2 dimensions, e.g. $\tr{A} \in \mathbb{R}^{L \times M \times N}$ \\ 
		$\tr{A}_{\star, 3, 2}$
					& Element of an array.
					The first index is the row, the second a column and the third a slice.
					"$\star$" selects all elements in that dimension.
					So in this case the full 3rd column of the second slice. \\ 
		$A^T, \vr{a}^T$
					& Transpose of $A$ and row vector $\vr{a}$ \\
		$i$			& Imaginary unit \\
		$\vr{x}, \vr{\omega}$
					& Spatial (time) domain and frequency domain coordinates \\
		$f(\vr{x})$
					& Function in spatial space \\
		$F(\vr{\omega})$
					& Function in Fourier space \\
		$F = \F f$
					& Fourier transform \\
		$f = \F^{-1} F$
					& Inverse Fourier transform \\
	\end{supertabular}
\end{center}
Zero based indices are used and summation ends at $N-1$.

\subsection*{Gaussian function and mixture model}
We define the multi dimensional Gaussian function as
\begin{align}
\label{eq:gaussian_definition}
g(\vr{x}, a, \vr{b}, C) = a e^{-\frac{1}{2}(\vr{x}-\vr{b})^TC(\vr{x}-\vr{b})},
\end{align}
where $a$ is the height, $\vr{b}$ the shift, $C$ the inverse of the shape (inverse of the covariance) of the Gaussian, and all parameters are real.
We use the inverse of the covariance in here, because the inversion is taken care of by our machine learning framework.
The factor $\frac{1}{2}$ and the inversion of $C$ are there to make convolution simpler.
This Gaussian would turn into a normal distribution if $a$ were replaced with the inverse of the integral of $e^{-\frac{1}{2}...}$.

A Gaussian mixture is defined as
\begin{align}
	m(\vr{x}, \vr{a}, B, \tr{C}) = \sum_{n=0}^{N} g(\vr{x}, \vr{a}_n, B_{\star, n}, C_{\star, \star, n}),
\end{align}
where $N$ is the number of Gaussians.
Our definition does not include any normalisation.
If desired, this could be included in $\vr{a}$.

\section{Theorems}
\begin{theorem}
	\label{th:der_xAx_respect_z}
	Let the scalar $\alpha$ be defined by the quadratic form
	\begin{align}
		\alpha = \vr{x}^T \mx{A} \vr{x},
	\end{align}
	where $\vr{x}$ is $n \times 1$, $\mx{A}$ is symmetric $n \times n$, and $\vr{x}$ is a function of the vector $\vr{z}$, while $\mx{A}$ does not depend on $\vr{z}$. Then
	\begin{align}
		\frac{\partial \alpha}{\partial \vr{z}} = 2 \vr{x}^T \mx{A}\frac{\partial \vr{x}}{\partial \vr{z}}.
	\end{align}
\end{theorem}

\begin{proof}
	See Barnes \cite{Barnes}, Proposition 14.
\end{proof}

\begin{theorem}
	\label{th:der_xAx_respect_A}
	Let the scalar $\alpha$ be defined by the quadratic form
	\begin{align}
	\alpha = \vr{x}^T \mx{A} \vr{x},
	\end{align}
	where $\vr{x}$ is $n \times 1$, and $\mx{A}$ is $n \times n$. Then
	\begin{align}
	\frac{\partial \alpha}{\partial \mx{C}} = \vr{x} \vr{x}^T.
	\end{align}
\end{theorem}

\begin{proof}
	\begin{align}
		\alpha &= \vr{x}^T \mx{A} \vr{x} \\
		&=
		\begin{pmatrix}
			x_1 & \ldots & x_n
		\end{pmatrix}
		\begin{pmatrix}
			c_{11} & \ldots & c_{1n} \\
			\vdots & \ddots & \vdots \\
			c_{n1} & \ldots & c_{nn}
		\end{pmatrix}
		\begin{pmatrix}
			x_1 \\
			\vdots \\
			x_n
		\end{pmatrix} = \sum_{i=1}^{n} \sum_{j=1}^{n} x_i x_j c_{ij}
	\end{align}
	The derivative of $\alpha$ with respect to element $c_{ij}$ in matrix $\mx{C}$ is $x_i x_j$.
	We can now construct a matrix with derivatives
	\begin{align}
		\frac{\partial \alpha}{\partial \mx{C}} =
		\begin{pmatrix}
			x_{1}x_{1} & \ldots & x_{1}x_{n} \\
			\vdots & \ddots & \vdots \\
			x_{n}x_{1} & \ldots & x_{n}x_{n}
		\end{pmatrix} = \vr{x} \vr{x}^T.
	\end{align}
\end{proof}

\section{Derivative of $g$ with respect to $\vr{x}$}
\begin{align}
	\frac{\partial}{\partial \vr{x}} g(\vr{x}, a, \vr{b}, C) =& \frac{\partial}{\partial \vr{x}} \left(a e^{-\frac{1}{2}(\vr{x}-\vr{b})^T C(\vr{x}-\vr{b})} \right) \nonumber \\
	=& a e^{-\frac{1}{2}(\vr{x}-\vr{b})^TC(\vr{x}-\vr{b})} \frac{\partial}{\partial \vr{x}} \left(-\frac{1}{2}(\vr{x}-\vr{b})^T C(\vr{x}-\vr{b})\right) \nonumber \\
	=& -\frac{a}{2} e^{-\frac{1}{2}(\vr{x}-\vr{b})^TC(\vr{x}-\vr{b})} \frac{\partial}{\partial \vr{x}} \left((\vr{x}-\vr{b})^T C(\vr{x}-\vr{b})\right) \nonumber \\
	=& -\frac{a}{2} e^{-\frac{1}{2}(\vr{x}-\vr{b})^TC(\vr{x}-\vr{b})} 2 (\vr{x}-\vr{b})^T C \frac{\partial (\vr{x}-\vr{b})}{\partial \vr{x}} \nonumber \\
	=& -a e^{-\frac{1}{2}(\vr{x}-\vr{b})^TC(\vr{x}-\vr{b})} (\vr{x}-\vr{b})^T C
\end{align}
Theorem \ref{th:der_xAx_respect_z} was used.

\section{Derivative of $g$ with respect to $a$}
\begin{align}
\frac{\partial}{\partial a} g(\vr{x}, a, \vr{b}, C) =& \frac{\partial}{\partial a} \left(a e^{-\frac{1}{2}(\vr{x}-\vr{b})^T C(\vr{x}-\vr{b})} \right) \nonumber \\
=& e^{-\frac{1}{2}(\vr{x}-\vr{b})^TC(\vr{x}-\vr{b})}
\end{align}

\section{Derivative of $g$ with respect to $\vr{b}$}
\begin{align}
\frac{\partial}{\partial \vr{b}} g(\vr{x}, a, \vr{b}, C) =& \frac{\partial}{\partial \vr{b}} \left(a e^{-\frac{1}{2}(\vr{x}-\vr{b})^T C(\vr{x}-\vr{b})} \right) \nonumber \\
=& a e^{-\frac{1}{2}(\vr{x}-\vr{b})^TC(\vr{x}-\vr{b})} \frac{\partial}{\partial \vr{b}} \left(-\frac{1}{2}(\vr{x}-\vr{b})^T C(\vr{x}-\vr{b})\right) \nonumber \\
=& -\frac{a}{2} e^{-\frac{1}{2}(\vr{x}-\vr{b})^TC(\vr{x}-\vr{b})} \frac{\partial}{\partial \vr{b}} \left((\vr{x}-\vr{b})^T C(\vr{x}-\vr{b})\right) \nonumber \\
=& -\frac{a}{2} e^{-\frac{1}{2}(\vr{x}-\vr{b})^TC(\vr{x}-\vr{b})} 2 (\vr{x}-\vr{b})^T C \frac{\partial (\vr{x}-\vr{b})}{\partial \vr{b}} \nonumber \\
=& a e^{-\frac{1}{2}(\vr{x}-\vr{b})^TC(\vr{x}-\vr{b})} (\vr{x}-\vr{b})^T C
\end{align}
Theorem \ref{th:der_xAx_respect_z} was used.
It is equal to the negative derivative with respect to $\vr{x}$


\section{Derivative of $g$ with respect to $\mx{C}$}
\begin{align}
\frac{\partial}{\partial \mx{C}} g(\vr{x}, a, \vr{b}, C) =& \frac{\partial}{\partial \mx{C}} \left(a e^{-\frac{1}{2}(\vr{x}-\vr{b})^T C(\vr{x}-\vr{b})} \right) \nonumber \\
=& a e^{-\frac{1}{2}(\vr{x}-\vr{b})^TC(\vr{x}-\vr{b})} \frac{\partial}{\partial \mx{C}} \left(-\frac{1}{2}(\vr{x}-\vr{b})^T C(\vr{x}-\vr{b})\right) \nonumber \\
=& -\frac{a}{2} e^{-\frac{1}{2}(\vr{x}-\vr{b})^TC(\vr{x}-\vr{b})} \frac{\partial}{\partial \mx{C}} \left((\vr{x}-\vr{b})^T C(\vr{x}-\vr{b})\right) \nonumber \\
=& -\frac{a}{2} e^{-\frac{1}{2}(\vr{x}-\vr{b})^TC(\vr{x}-\vr{b})} (\vr{x}-\vr{b}) (\vr{x}-\vr{b})^T
\end{align}
Theorem \ref{th:der_xAx_respect_A} was used.

\section{Derivatives of the Gaussian mixture $m$}
The derivative with respect to $\vr{x}$ is simply the sum of the derivatives for the components.

The derivatives for the parameters of the components can be computed separately with the equations above.


\begin{thebibliography}{9}% 2nd arg is the width of the widest label.
	\bibitem{Barnes} Barnes, Randal J,  Matrix differentiation (and some other stuff) (2006)
\end{thebibliography}

\end{document}
